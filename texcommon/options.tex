% LaTeX options for the informatics course given
% at HES-SO Valais // Wallis
%
% @author Pierre-André Mudry, pierre-andre.mudry@hevs.ch
% @version 1.9
% @date October 2009
% @revision June 2010
% @revision September 2010
% @revision January 2014, new shadow box for lstlisting using mdframed
% @revision November 2018, watermark as option
% @revision October 2022, autogobble
% @revision August 2023, making all this presentable for my colleaguess

% Some useful definitions
\def\defaultauthor{mui}
\def\authorfullname{Dr Pierre-Andr\'{e} Mudry}
\def\institution{Filière ISC - Haute Ecole d'Ingénierie, Sion}
\def\coursename{101 - Fondements de la programmation 1}
\def\givenyears{2022-2023}

\usepackage[T1]{fontenc} % Encoding of fonts
\usepackage[utf8]{inputenc} % Encoding of data input
\usepackage[french]{babel} % This is the recommended version for French
% \usepackage[french, german]{babel} % This is the recommended version for French

\usepackage{textcomp, gensymb}

\usepackage[svgnames]{xcolor}
\definecolor{shadecolor}{named}{Yellow} % Set the color of the shaded environment, used by todo command
\usepackage{fourier-orns} % For the symbols such as warning, bomb etc.. Like fourier but without messing with math fonts
\usepackage{listings}
\usepackage{lstautogobble} % Removes indentations at the beginning of lstlistings
\usepackage{geometry,pdfpages} % Geometry replaced vmargin because it is more flexible and handlge double pages
\usepackage{url}
\usepackage{hyperref}
\usepackage{graphicx}
\usepackage{subfig}
\usepackage{ifthen}
\usepackage{enumerate}
\usepackage{amssymb, amsmath}
\usepackage{booktabs} % Nicer looking tables
\usepackage{marvosym} % For the nice mouse symbol
\usepackage{algorithm}
\usepackage{algorithmic}
\usepackage{enumerate}
\usepackage{colortbl}
\usepackage{multirow}
\usepackage{multicol}
\usepackage{fancybox}
\usepackage{dashrule} % For lines with dashes (true/false)

\usepackage{fontawesome}

% https://www.ctan.org/tex-archive/fonts/sourcesanspro/
\usepackage[default, lining]{sourcesanspro}
  

%\usepackage{microtype}
%\usepackage[protrusion=true,expansion]{microtype}

%  Use \def\confidential{} to make the confidential mark appear

\ifdefined\confidential
  \usepackage{draftwatermark}
  \SetWatermarkScale{0.5}
  \SetWatermarkLightness{0.9}
  \SetWatermarkText{CONFIDENTIAL}
\else
\fi

% Examples for pictures anywhere

%\begin{tikzpicture}[remember picture,overlay]
  %\node[anchor=south west,inner sep=0pt] at ($(current page.south west)+(4.5cm,4.5cm)$) {
     %\includegraphics[width=2cm]{figs/rabbit.png}
  %};
%\end{tikzpicture}

%\end{parts}
%\begin{tikzpicture}[remember picture,overlay]
  %\node[anchor=south east,inner sep=0pt] at ($(current page.south east)+(0.8cm,4.5cm)$) {
     %\includegraphics[width=1.5cm]{figs/pokeball.png}
  %};
%\end{tikzpicture}


%%%%%%%%%%%%
% Font related
%%%%%%%%%%%%
%\usepackage[protrusion=true, expansion]{microtype}
%\usepackage{palatino}
% OK fira is nice but it has ligatures and on paper beramono is better
\usepackage[scaled=0.85]{beramono}

% We use firamono with a medium version of the bold
% see https://ctan.org/pkg/fira
%\usepackage[medium, scaled=0.85]{FiraMono}

%\usepackage[defaultmono,scale=0.1]{droidmono}
%\usepackage{inconsolata} % Used to replace texttt font

% For remarkbox
\usepackage{tikz}
\usetikzlibrary{calc,shapes,shadows,decorations,positioning}

\renewcommand{\algorithmiccomment}[1]{// \emph{#1}}

% Set margins dimensions to something more reasonable
% \setpapersize{A4} \setmargrb{22mm}{14mm}{22mm}{16mm}

\ifdefined\exam
% For exams only
\geometry{
  a4paper,twoside=true,bindingoffset=5mm, %showframe,
  left=15mm,top=10mm,right=15mm,bottom=15mm,
  headheight=6mm,headsep=7mm,foot=10mm,footskip=13mm,
  includeheadfoot
}
\else
% For series only
\geometry{
  a4paper, %showframe,
  left=20mm,top=16mm,right=20mm,bottom=16mm,
  headheight=6mm,headsep=7mm,foot=10mm,footskip=13mm,
  includeheadfoot
}
\fi

% \captionsetup[figure]{name=Figure}

%%%%%%%%%%%%%%%%%%%%%%%%%%%%%%%%%%%%%%%%%%%%%%%%%%%%%%%%%%%%%%%%%%%%%%%%%%
% Options for the listing package, prints code in a nice gray box
% %%%%%%%%%%%%%%%%%%%%%%%%%%%%%%%%%%%%%%%%%%%%%%%%%%%%%%%%%%%%%%%%%%%%%%%%
%
% general listing colors
%
\definecolor{listing-rule}{HTML}{3e3e3e}
\definecolor{listing-numbers}{HTML}{B3B2B3}
\definecolor{listing-text-color}{HTML}{000000}
\definecolor{listing-keyword}{HTML}{0033B3}
\definecolor{listing-keyword-2}{HTML}{00627A} % additional keywords
\definecolor{listing-keyword-3}{HTML}{9137CB} % additional keywords
\definecolor{listing-identifier}{HTML}{000000}
\definecolor{listing-string}{HTML}{067D17}
\definecolor{listing-comment}{HTML}{8C8C8C}

% La couleur du texte verbatim en quote
\definecolor{hei}{HTML}{d41367}
\definecolor{hei_dark}{HTML}{000000}
%\definecolor{hei_dark}{HTML}{393939}

\lstdefinestyle{basic_style}{    
  language     = ,
  upquote      = true,
  breaklines   = false,
  tabsize      = 2,
  basicstyle   = {\ttfamily\color{hei_dark}\bfseries},
  commentstyle = {\rmfamily\textit\color{hei_dark}\bfseries},
  stringstyle  = {\ttfamily\color{hei_dark}\bfseries},
  keywordstyle = {\ttfamily\bfseries\color{hei_dark}\bfseries},
  showstringspaces=false,
  escapeinside     = {(*}{*)}, % Allow LaTeX inside these special comments
  literate         =
  {á}{{\'a}}1 {é}{{\'e}}1 {í}{{\'i}}1 {ó}{{\'o}}1 {ú}{{\'u}}1
  {Á}{{\'A}}1 {É}{{\'E}}1 {Í}{{\'I}}1 {Ó}{{\'O}}1 {Ú}{{\'U}}1
  {à}{{\`a}}1 {è}{{\'e}}1 {ì}{{\`i}}1 {ò}{{\`o}}1 {ù}{{\`u}}1
  {À}{{\`A}}1 {È}{{\'E}}1 {Ì}{{\`I}}1 {Ò}{{\`O}}1 {Ù}{{\`U}}1
  {ä}{{\"a}}1 {ë}{{\"e}}1 {ï}{{\"i}}1 {ö}{{\"o}}1 {ü}{{\"u}}1
  {Ä}{{\"A}}1 {Ë}{{\"E}}1 {Ï}{{\"I}}1 {Ö}{{\"O}}1 {Ü}{{\"U}}1
  {â}{{\^a}}1 {ê}{{\^e}}1 {î}{{\^i}}1 {ô}{{\^o}}1 {û}{{\^u}}1
  {Â}{{\^A}}1 {Ê}{{\^E}}1 {Î}{{\^I}}1 {Ô}{{\^O}}1 {Û}{{\^U}}1
  {œ}{{\oe}}1 {Œ}{{\OE}}1 {æ}{{\ae}}1 {Æ}{{\AE}}1 {ß}{{\ss}}1
  {ç}{{\c c}}1 {Ç}{{\c C}}1 {ø}{{\o}}1 {å}{{\r a}}1 {Å}{{\r A}}1
  {€}{{\EUR}}1 {£}{{\pounds}}1 {«}{{\guillemotleft}}1
  {»}{{\guillemotright}}1 {ñ}{{\~n}}1 {Ñ}{{\~N}}1 {¿}{{?`}}1
  {…}{{\ldots}}1 {≥}{{>=}}1 {≤}{{<=}}1 {„}{{\glqq}}1 {“}{{\grqq}}1
  {”}{{''}}1
}

% So numbers are not copied
\usepackage{accsupp}
\newcommand{\noncopynumber}[1]{%
    \BeginAccSupp{method=escape,ActualText={}}%
    #1%
    \EndAccSupp{}%
}

\lstdefinestyle{framed}{
    style=basic_style,
    basicstyle={\ttfamily\color{hei_dark}},
    % backgroundcolor  = \color{listing-background},    
    breaklines       = true,      
    % frame            = single,
    % framesep         = 0.5em,
    rulecolor        = \color{listing-rule},
    % frameround       = tttt,
    tabsize          = 4,
    numberstyle      = {\footnotesize \color{listing-numbers}\noncopynumber},
    % aboveskip        = 1.0em,
    % belowskip        = 0.1em,
    % abovecaptionskip = 0em,
    % belowcaptionskip = 1.0em
}

% This is the default style when no language has been selected
\lstset{style=framed}

%
% Java (Java SE 12, 2019-06-22)
%
\lstdefinelanguage{Java}{
  style=framed,
  morekeywords={
    % normal keywords (without data types)
    abstract,assert,break,case,catch,class,continue,default,
    do,else,enum,exports,extends,final,finally,for,if,implements,
    import,instanceof,interface,module,native,new,package,private,
    protected,public,requires,return,static,strictfp,super,switch,
    synchronized,this,throw,throws,transient,try,volatile,while,
    % var is an identifier
    var
  },
  morekeywords={[2] % data types
    % primitive data types
    boolean,byte,char,double,float,int,long,short,
    % String
    String,
    % primitive wrapper types
    Boolean,Byte,Character,Double,Float,Integer,Long,Short
    % number types
    Number,AtomicInteger,AtomicLong,BigDecimal,BigInteger,DoubleAccumulator,DoubleAdder,LongAccumulator,LongAdder,Short,
    % other
    Object,Void,void
  },
  morekeywords={[3] % literals
    % reserved words for literal values
    null,true,false,
  },
  sensitive,
  morecomment  = [l]//,
  morecomment  = [s]{/*}{*/},
  morecomment  = [s]{/**}{*/},
  morestring   = [b]",
  morestring   = [b]',
}

\lstdefinelanguage{scala}{
  style=framed,
  basicstyle={\ttfamily\color{hei_dark}},
  morekeywords={%
          abstract,case,catch,class,def,do,else,extends,%
          false,final,finally,for,forSome,if,implicit,import,lazy,%
          match,new,null,object,override,package,private,protected,%
          return,sealed,super,this,throw,trait,true,try,type,%
          val,var,while,with,yield},
  otherkeywords={=>,<-,<\%,<:,>:,\#,@},
  sensitive=true,
  upquote = true,
  autogobble = true,
  morecomment=[l]{//},
  morecomment=[n]{/*}{*/},
  morestring=[b]",
  morestring=[b]',
  morestring=[b]""",
  keywordstyle     = {\color{listing-keyword}},
  keywordstyle     = {[2]\color{listing-keyword-2}\bfseries},
  keywordstyle     = {[3]\color{listing-keyword-3}\bfseries\itshape},
  identifierstyle  = \color{listing-identifier},
  commentstyle     = {\color{listing-comment}\itshape},
  stringstyle      = \color{listing-string}
}[keywords,comments,strings]


% Let mdframed do the frame (to remove the border that goes too far)
\newcommand*\continuingfont{\scriptsize\itshape}
\newcommand*\continuingtext{Listing continues on next page...}
\newcommand*\continuedtext{\ldots continuing from previous page}

\usepackage[%
    framemethod=tikz,
    skipbelow=\topskip,
    skipabove=\topskip
]{mdframed}

\mdfsetup{%
    usetwoside=false,
    skipabove=9pt, % Removes space before the block
    skipbelow=0pt, % Removes space after the bloc
    % leftmargin=0pt,
    % rightmargin=0pt,
    innertopmargin=0pt,
    innerbottommargin=-2pt,    
    shadow=false,    
    backgroundcolor=black!5, % 5 percent blackness for the background
    align=center,    
    middlelinecolor=black!75,
    firstextra = { % Should put a small text if the listing does not fit a single page
      \node[below left,overlay,align=right,font=\continuingfont]
        at (O -| P) {\continuingtext};
    } ,
    secondextra  = { % Starting a new page with a listing
      \node[above right,overlay,align=left,font=\continuingfont]
        at (O |- P) {\continuedtext};
    } ,
    middleextra  = { % Used when spanning multiple pages (3)
      \node[below right,overlay,align=left,font=\continuingfont]
        at (O) {\continuingtext};
      \node[above right,overlay,align=left,font=\continuingfont]
        at (O |- P) {\continuedtext};
    },
    roundcorner=4
}

\usepackage{etoolbox}% >= v2.1 2011-01-03

\BeforeBeginEnvironment{lstlisting}{\begin{mdframed}\vspace{-0.em}}
\AfterEndEnvironment{lstlisting}{\vspace{-0.6em}\end{mdframed}}

\lstdefinestyle{with_frame}{language=scala, breaklines, numbers = none, extendedchars = true, mathescape=false, captionpos=b}

\lstdefinestyle{small_frame}{style=with_frame, basicstyle=\scriptsize\ttfamily}

\lstdefinestyle{numbered_frame}{style=with_frame, numbers=left, numberstyle=\scriptsize, stepnumber=1, numbersep = 15pt}

\lstdefinestyle{small_numbered_frame}{style=numbered_frame, basicstyle=\scriptsize\ttfamily}

\lstdefinestyle{verbatim_io}{tabsize = 6,autogobble = true, tab=\rightarrowfill, breaklines, extendedchars = true, mathescape = false, captionpos=b, stringstyle=\ttfamily}

\lstdefinestyle{verbatim_io_small}{style=verbatim_io, basicstyle=\scriptsize\ttfamily}

\lstdefinestyle{verbatim_io_numbered}{style=verbatim_io, numbers=left, numberstyle=\scriptsize, stepnumber=1, numbersep=15 pt}

\lstdefinestyle{verbatim_with_tabs}{style=verbatim_io_numbered, showtabs=true, showspaces = true, tab=\rightarrowfill}

% Defines new environment, faster --> \begin{java} \end{java}
\lstnewenvironment{scala}
{\lstset{style=numbered_frame}}
{}

\lstnewenvironment{small_scala}
{\lstset{style=small_numbered_frame}}
{}

\lstnewenvironment{verbatim_lst}
{\lstset{style=verbatim_io}}
{}

\lstnewenvironment{verbatim_lst_small}
{\lstset{style=verbatim_io_small}}
{}

\lstnewenvironment{verbatim_tabs}
{\lstset{style=verbatim_with_tabs}}
{}

% % Required to add the mdframed before and after the listings
\surroundwithmdframed{lstlisting}
\surroundwithmdframed{scala}
\surroundwithmdframed{verbatim_tabs}
\surroundwithmdframed{verbatim_lst}


%%%%%%%%%%%%% This is the previous version from INF1
% \lstset{breaklines = false, tabsize=2, basicstyle=\small\ttfamily, commentstyle = \rmfamily\textit, stringstyle=\ttfamily, keywordstyle=\ttfamily\bfseries,showstringspaces=false}

% \lstdefinestyle{verbatim_io}{tabsize = 6, tab=\rightarrowfill, breaklines, extendedchars = true, mathescape = true, captionpos=b, stringstyle=\ttfamily}
% \lstdefinestyle{verbatim_io_small}{style=verbatim_io, basicstyle=\scriptsize\ttfamily}
% \lstdefinestyle{verbatim_io_numbered}{style=verbatim_io, numbers=left, numberstyle=\scriptsize, stepnumber=1, numbersep=15 pt}
% \lstdefinestyle{verbatim_with_tabs}{style=verbatim_io_numbered, showtabs=true, showspaces = true, tab=\rightarrowfill}

% % Java styles
% \lstdefinestyle{with_frame}{language= Java, breaklines, extendedchars = true, mathescape = true, captionpos=b}
% \lstdefinestyle{small_frame}{style=with_frame, basicstyle=\scriptsize\ttfamily}
% \lstdefinestyle{numbered_frame}{style=with_frame, numbers=left, numberstyle=\scriptsize, stepnumber=1, numbersep = 15pt}
% \lstdefinestyle{small_numbered_frame}{style=numbered_frame, basicstyle=\scriptsize\ttfamily}

% % Defines new environment, faster --> \begin{java} \end{java}
% \lstnewenvironment{java}
% {\lstset{style=numbered_frame}}
% {}

% \lstnewenvironment{small_java}
% {\lstset{style=small_numbered_frame}}
% {}

% \lstnewenvironment{verbatim_lst}
% {\lstset{style=verbatim_io}}
% {}

% \lstnewenvironment{verbatim_lst_small}
% {\lstset{style=verbatim_io_small}}
% {}

% \lstnewenvironment{verbatim_tabs}
% {\lstset{style=verbatim_with_tabs}}
% {}

% % Let mdframed do the frame (to remove the border that goes too far)
% \newcommand*\continuingfont{\scriptsize\itshape}
% \newcommand*\continuingtext{Listing continues on next page $\rightarrow$}
% \newcommand*\continuedtext{\ldots continuing from previous page}
% \usepackage[%
%     framemethod=tikz,
%     skipbelow=0,
%     skipabove=\topskip
% ]{mdframed}
% \mdfsetup{%
%     usetwoside=false,
%     leftmargin=0pt,
%     rightmargin=0pt,
%     innertopmargin=0pt,
%     innerbottommargin=-2.5pt,
%     backgroundcolor=black!5, % 5 percent blackness for the background
%     middlelinecolor=black,
%     firstextra = { % Should put a small text if the listing does not fit a single page
%       \node[below left,overlay,align=right,font=\continuingfont]
%         at (O -| P) {\continuingtext};
%     } ,
%     secondextra  = { % Starting a new page with a listing
%       \node[above right,overlay,align=left,font=\continuingfont]
%         at (O |- P) {\continuedtext};
%     } ,
%     middleextra  = { % Used when spanning multiple pages (3)
%       \node[below right,overlay,align=left,font=\continuingfont]
%         at (O) {\continuingtext};
%       \node[above right,overlay,align=left,font=\continuingfont]
%         at (O |- P) {\continuedtext};
%     },
%     roundcorner=4
% }

% %Added to make the lines numbers in listings displayable, otherwise it is not shown because
% %mdframed does not take the correct thing
% \makeatletter
% \renewrobustcmd*\mdf@tikzbox@tfl[1]{%three or four borders
%   \path(0,0)rectangle(\mdfboundingboxwidth,\mdfboundingboxheight);% replace \clip by \path
%     \begin{scope}[mdfcorners]%
%       \path
%         [preaction=mdfouterline]%
%         [postaction=mdfbackground]%
%         [postaction=mdfinnerline]#1;%
%     \end{scope}%
%   \path[mdfmiddleline,mdfcorners]#1;
% }%

% % Required to add the mdframed before and after the listings
% \surroundwithmdframed{lstlisting}
% \surroundwithmdframed{java}
% \surroundwithmdframed{small_java}
% \surroundwithmdframed{verbatim_tabs}
% \surroundwithmdframed{verbatim_lst}

% %%%%%%%%%%%%%%%%%%%%%%%%%%%%%%%%%%%%%%%%%%%%%%%%%%%%%%%%%%%%%%%%%%%%%%%%
% Automatic links in PDF files and default open options
% %%%%%%%%%%%%%%%%%%%%%%%%%%%%%%%%%%%%%%%%%%%%%%%%%%%%%%%%%%%%%%%%%%%%%%%%
\hypersetup{colorlinks, %
citecolor=black,%
filecolor=black,%
linkcolor=black,%
urlcolor=blue,%
pdfauthor={\authorfullname, \institution},
pdfkeywords={\coursename, exercises},
pdfpagelayout=SinglePage,
bookmarksopen = true,
pdfstartview=Fit,
bookmarksopenlevel = 0,
pdftitle=\thetitle
}

% Add a new command for things that are not done yet
%\newcommand{\todo}[1]{\begin{shaded}[\textsc{TODO}: \emph{{#1}}]\end{shaded}}
\newcommand{\todo}[1]{\colorbox{Yellow}{\textsc{TODO}: \emph{{#1}}}}
\newcommand{\colored}[1]{\colorbox{VioletRed}{{#1}}}

% Redefine Solution, coz we need both french and german version.
\renewcommand{\solutiontitle}{\noindent\textbf{Solution:}\enspace}

% Redefine sectioning, coz we don't like it like this
\newcounter{sec_count}
\renewcommand{\section}[1]{\addtocounter{sec_count}{1}\fullwidth{\Large \textbf{\emph{Part \arabic{sec_count} - #1}}}}

% Header and footer configuration
\headrule
\header{\small{\shorttitle}}{}{\small{Rev.~\rev{}}}

\footrule
%\includegraphics[width=2.2cm]{\path FR-DE_HEI.pdf}
\footer{\small{Dr Pierre-André Mudry $\mid$ \emph{101.1 Programmation impérative}}}{}{\small{\thepage$\mid$\numpages}}

%%%%%%%%%%%%%%%%%%%%%%%%%%
%% Exam class related
%%%%%%%%%%%%%%%%%%%%%%%%%%
% Format for the questions
\def\qtext{\textbf{Question \thequestion}}
%\def\qtext{\textbf{Question $\mid$ Frage \thequestion}}
\qformat{\large{\qtext}\hfill}

% Define new question command, named cquestion, that prints a small keyboard for questions that are supposed to be done on computer
\newcommand{\standardqformat}{\qformat{\large{\textbf{Question \thequestiontitle}}\hfill}}
\newcommand{\computerqformat}{\qformat{\large{\textbf{Question \thequestiontitle} \huge \ComputerMouse}\hfill}}
\newcommand{\computertqformat}[1]{\qformat{\large{\textbf{Question \thequestiontitle} {\huge \ComputerMouse} -- \emph{#1}}\hfill}}
\newcommand{\titleformat}[1]{\qformat{\large{\textbf{Question \thequestiontitle~-- \emph{#1}}}\hfill}}
\newcommand{\cquestion}{\computerqformat\question\standardqformat}
\newcommand{\tcquestion}[1]{\computertqformat{#1}\question\standardqformat}
\newcommand{\tquestion}[1]{\titleformat{#1}\question\standardqformat}

%\newcommand{\standardqformat}{\qformat{\large{\qtext\hfill}}
%\newcommand{\computerqformat}{\qformat{\large{\qtext} -- \huge \ComputerMouse}\hfill}}
%\newcommand{\cquestion}{\computerqformat\question\standardqformat}

% Renew subpartnumbering
\renewcommand\thesubpart{\arabic{subpart}}
\renewcommand\subpartlabel{\thesubpart)}

%%%%%%%%%%%%%%%%%%%%%%%%%%%%%%%
% Allow to ask a true/false question that displays nicely
%%%%%%%%%%%%%%%%%%%%%%%%%%%%%%%
\newenvironment{tightcenter}{%
  \addtolength\partopsep{-1.5cm}
  \setlength\parskip{0pt}
  \begin{center}
}{%
  \end{center}
}

\newcommand{\begintruefalse}{
  %\begin{tightcenter}{\hdashrule[0.50ex]{\linewidth}{0.5pt}{1pt 1pt}}\end{tightcenter}
  \begin{tightcenter}\dotfill\end{tightcenter}\null
  \vspace{-1mm}
}

\newcommand{\truefalse}[2]{
\vspace{-2.5mm}
\begin{minipage}{0.8\linewidth}
{#1}
\end{minipage}
\quad
\ifprintanswers{
  \ifthenelse{\equal{#2}{true}}
  	{
  	  \begin{minipage}{2.5cm}\raggedleft
        \begin{tabular}{c|c}
            \emph{True} & \emph{False} \\
            $\otimes$ & $\Box$\\
        \end{tabular}
        \end{minipage}
  	}{}
  \ifthenelse{\equal{#2}{false}}
  	{
  	  \begin{minipage}{2.5cm}\raggedleft
        \begin{tabular}{c|c}
        \emph{True} & \emph{False} \\
        $\Box$ & $\otimes$\\
        \end{tabular}
	  \end{minipage}
  	}{}
    %\begin{tightcenter}{\hdashrule[0.50ex]{\linewidth}{0.5pt}{1pt 1pt}}\end{tightcenter}
    \begin{tightcenter}\dotfill\end{tightcenter}\null        
}
\else
{
  \begin{minipage}{2.5cm}\raggedleft
  \begin{tabular}{c|c}
  \emph{True} & \emph{False} \\
  $\Box$ & $\Box$\\
  \end{tabular}
  \end{minipage}
  \vspace{0.9mm}
  %\begin{tightcenter}{\hdashrule[0mm]{\linewidth}{0.5pt}{1.5pt}}\end{tightcenter}
  %\begin{tightcenter}{\hdashrule[0.50ex]{\linewidth}{0.5pt}{1pt 1pt}}\end{tightcenter}
  \begin{tightcenter}\dotfill\end{tightcenter}\null
  %\vspace{1.5mm}
}
\fi
}

%%%%%%%%%%%%%%%%%%%%%%%%%%%%%%%
% answerli, like answerline but on a single line
% (which takes less space)
%%%%%%%%%%%%%%%%%%%%%%%%%%%%%%%
\newcommand{\answerli}[2]{
\begin{minipage}{0.65\columnwidth}
{#1}
\end{minipage}
\quad
\ifprintanswers{
  \begin{minipage}{0.16\columnwidth}
	{#2}
  \end{minipage}
}
\else
{
  \begin{minipage}{0.16\columnwidth}
	\hrulefill
  \end{minipage}
}
\fi
}


%%%%%%%%%%%%%%%%%%%%%%%%%%%%%%%
% A line to separate german and french versions
%%%%%%%%%%%%%%%%%%%%%%%%%%%%%%%
\newcommand{\lineSep}{
\begin{center}
\vspace{-1mm}
\makebox[6cm][c]{\hrulefill}
\vspace{1.5mm}
\end{center}
}

%%%%%%%%%%%%%%%%%%%%%%%%%%%%%%%
% Lookout, there's more to come
%%%%%%%%%%%%%%%%%%%%%%%%%%%%%%%
\newcommand{\turnWarning}{
\vfill{}
\begin{flushright}\emph{\warning Turn page $\rightarrow$} \end{flushright}
\newpage
}

%%%%%%%%%%%%%%%%%%%%%%%%%%%%%%%
% Creates an empty page and say it was left empty on purpose
%%%%%%%%%%%%%%%%%%%%%%%%%%%%%%%
\newcommand{\leerseite}{
\newpage
\vspace*{\fill}
\hrule
\begin{center}
{\Large \emph{Leerseite}\\\vspace{-1.5mm}$\diamond$\vspace{1mm}\\Cette page a été laissée vierge intentionnellement}
\end{center}
\hrule
\vspace{\fill}
\newpage
}

%%%%%%%%%%%%%%%%%%%%%%%%%%%%%%%
% Last page filler if required
%%%%%%%%%%%%%%%%%%%%%%%%%%%%%%%
\newcommand{\lastPage}{
\vspace*{\fill}
\hrule
\begin{center}
\Large \textit{Fin$\mid$Ende}\\
\end{center}
\hrule
\vspace{\fill}
}


%%%%%%%%%%%%%%%%%%%%%%%%
%% Remark box
%%%%%%%%%%%%%%%%%%%%%%%%
% New implementation of the remark box, using TikZ
\newcommand{\remarkbox}[1]{%
         \begin{center}%
            \begin{tikzpicture}%
% Title
                \node (a) [rectangle, draw=black, top color=gray!10, bottom color=gray!20, rounded corners=5pt, inner xsep=8pt, inner ysep=8pt, outer ysep=10pt]{
                \begin{minipage}{0.75\linewidth}#1\end{minipage}};%
			\node[rectangle, rounded corners=5pt, inner xsep=6pt, inner ysep=6pt, above=0.5mm, fill=white, draw=black] at (a.330) {\textsc{Remark}};
            \end{tikzpicture}%
         \end{center}%
}

%%%%%%%%%%%%%%%%%%%%%%%%%%%%%%%
%% Title box for the exam
%%%%%%%%%%%%%%%%%%%%%%%%%%%%%%%
\newcommand{\titlebox}[1]{%
         \begin{center}%
            \begin{tikzpicture}%
% Title
                \node (a) [rectangle, drop shadow, fill=white, draw=black, rounded corners=5pt, inner xsep=8pt, inner ysep=8pt, outer ysep=10pt]{
                \begin{minipage}{0.85\linewidth}#1\end{minipage}};%
            \end{tikzpicture}%
         \end{center}%
}

%%%%%%%%%%%%%%%%%%%%%%%%%%%%%%%
% For turnpage warning symbol
%%%%%%%%%%%%%%%%%%%%%%%%%%%%%%%
\newcommand{\turnpage}{\vfill{}\begin{flushright}\emph{Turn page $\rightarrow$} \end{flushright}}

%%%%%%%%%%%%%%%%%%%%%%%%%%%%%%%
% BigO notation
%%%%%%%%%%%%%%%%%%%%%%%%%%%%%%%
\newcommand{\bigO}[1]{\mathcal{O}\mathopen{}\left(#1\right)}